\documentclass{scrartcl}
\usepackage[a4paper, total={7in, 10in}]{geometry}
\usepackage{stmaryrd}
\usepackage{amsthm}
\usepackage{amssymb}
\usepackage{amsmath}
\usepackage{algorithm2e}
\usepackage{hyperref}
\usepackage[french]{babel}
\usepackage{color}
\usepackage{tikz}
\usetikzlibrary{automata, arrows.meta, positioning}
\usepackage{MacrosArt}


\begin{document}

\title{Stage Map Graphs}

\author{José Lorgeré}

\maketitle

\begin{flushleft}

\section*{Introduction}

\section{Trucs généraux}

\subsection{Résultats et définitions préliminaires}

\begin{def*}[Carte]
    Une carte est une fonction $f : V \rightarrow \mathcal{P}(\mathbb{S}^2)$ telle que pour tout $v \in V$, $f(v)$
    est homéomorphe à $\mathbb{D}^2$ et telle que pour $v \neq u$, $f(u)$ et $f(v)$ sont d'intérieur disjoints. Si
    $f$ forme un recouvrement de $\mathbb{S}^2$, on la dira sans trou, ou complète. On appelera les $f(v)$ régions
\end{def*}

\begin{def*}[Graphe de carte]
    Un graphe $G = (V, E)$ est de carte s'il existe une carte $f$ sur $V$ telle que $xy \in E$ si et seulement si
    $f(x) \cap f(y) \neq \varnothing$
\end{def*}

\begin{theorem}[Caractérisation des graphes de carte]\label{carCarte}
    Un graphe $G = (V, E)$ est de carte si et seulement si il existe $H$ un graphe biparti planaire dont un des
    côtés de la bipartition est $V$, tel que $H^2[V] = G$. Un tel graphe $H$ est appelé graphe témoin de $G$
\end{theorem}

\begin{prop}
    Si $G$ est un graphe de carte, et $H = G[A]$ où $A \subset V$, alors $H$ est un graphe de carte
\end{prop}

\begin{proof}
    On reprend la carte de $G$ où l'on ne garde que les régions identifées aux sommets présents dans $A$
\end{proof}

\begin{def*}[Join]
    Le join de $2$ graphes $G = (V, E)$ et $G' = (V', E')$ est le graphe ayant pour sommet $V'' = V \cup V'$ et pour arrêtes
    \[ E'' = E \cup E' \cup \{ xx' \mid x \in V, x \in V' \} \]
    On notera pour tout $n \geq 1$ $K(n, G)$ le join d'un indépendant à $n$ sommets avec le graphe $G$
\end{def*}

\begin{prop}
    Soit $G$ un graphe à $3$ sommets. $K(3, G)$ n'est pas un graphe de carte
\end{prop}

\begin{proof}
    Supposons par l'absurde que $K(3, G)$ est un graphe de carte. On se donne alors $H$ vérifiant toutes les propriétés
    du théorème \ref{carCarte}. On note pour tout $v_i$, sommet indépendant de $K(3, G)$, $N_i$ son voisinage dans $H$
    incluant $v_i$. On note $H'$ le graphe $(((H / N_1) / N_2) / N_3)$, qui est donc un mineur de $H$ et donc planaire.
    Or $H' = K_{3,3}$ : en effet $N_i$ est adjacent à tout les sommets de $G$, comme le voisinage de $v_i$ dans $H$ l'est,
    car $v_i$ l'est dans $K(3, G)$. De plus il n'y a pas d'arrêtes entre les sommets de $G$ comme $H$ est biparti, et pas
    d'arrêtes entre les différents $N_i$ comme les $v_i$ sont indépendants dans $G$.\\
    Absurde, $K(3, G)$ n'est pas un graphe de carte
\end{proof}

\subsection{Contractibilité des arrêtes}

On aimerait donner une condition nécessaire et suffisante permettant de contracter les arrêtes d'un graphe de carte.
Certaines arrêtes sont clairement contractibles, comme celles représentées par $2$ régions partageant une courbe, d'autres ne le sont
pas, comme l'arrête $xy$ du graphe $G_{pch}$ \ref{Gpch}, qui est un graphe de carte.
On dénotera par $\mathcal{G}_{pch}$ l'ensemble des graphes obtenu depuis $G_{pch}$,
en ajoutant des arrêtes entre les sommets distincts de $a, b, c$

\begin{figure}[h]
    \caption{Le graphe $G_{pch}$}\label{Gpch}
    \begin{center}
        \begin{tikzpicture}[auto]
            \begin{scope}[every node/.style={circle, draw}]
                \node (a) {$a$};
                \node (b) [right = 20mm of a] {$b$};
                \node (c) [right = of b] {$c$};
                \node (x) [below left = 12mm and 5mm of a] {$x$};
                \node (y) [below right = 12mm and 5mm of a] {$y$};
                \node (u) [below = of b] {$u$};
                \node (v) [below = of c] {$v$};
            \end{scope}

            \path
            (a) edge (x) edge (y) edge (u) edge (v)
            (b) edge (x) edge (u) edge (v)
            (c) edge (y) edge (u) edge (v)
            (x) edge (y) edge[bend right] (u)
            (y) edge (u);
        \end{tikzpicture}

    \end{center}
\end{figure}

Il sera utile de savoir contracter certaines arrêtes particulières des graphes de carte. Si $G$ est un graphe de carte
et $H$ un témoin de $G$, si l'arrête $xy$ est représentée par une des $2$ configurations de la figure \ref{IzEdge}, alors
on peut la contracter en maintenant $G$ de carte : pour la première il s'agit de deux contractions d'arrêtes dans un graphe
planaire, pour la deuxième, il suffit d'ajouter une arrête subdivisée entre $x$ et $y$ passant par l'intérieur de la face à laquelle
ces deux derniers sont adjacents pour obtenir un témoin de $G$ présentant le premier cas

\begin{figure}[h]
    \caption{Cas d'arrêtes dont la contraction laisse $G$ de carte}\label{IzEdge}
    \begin{center}
        Le deuxième cas se généralise à des cliques arbitrairement grandes, dès que $x$ et $y$ sont adjacents à une même face
        \\~\\
        \begin{tikzpicture}[auto]
            \begin{scope}[every node/.style={circle, draw}]
                \node (x1) {$x$};
                \node (x2) [right = 50mm of x1] {$x$};
            \end{scope}

            \begin{scope}[every node/.style={rectangle, draw, fill=black}]
                \node (u) [right = of x1] {$\0$};
                \node (v) [right = of x2] {$\0$};
            \end{scope}

            \begin{scope}[every node/.style={circle, draw}]
                \node (y1) [right = of u] {$y$};
                \node (y2) [above = of v] {$y$};
                \node (z) [below = of v] {$z$};
                \node (t) [right = of v] {$t$};
            \end{scope}

            \path
            (u) edge (x1) edge (y1)
            (v) edge (x2) edge (y2) edge (z) edge (t);
        \end{tikzpicture}

    \end{center}
\end{figure}

\begin{def*}[Arrête pincée]
    On appellera arrête pincée de $G$ toute arrête $xy$ telle qu'il existe un sous graphe induit de $G$ contenant $xy$ se contractant en
    $G_p \in \mathcal{G}_{pch}$, et telle que la contraction n'exploite pas les arrêtes du sous graphe induit par $x,y,a,u$
\end{def*}

Les conditions sur les arrêtes à contracter ne sont pas strictement nécessaires mais rendent plus agréables les preuves

\begin{prop}
    Soit $G$ un graphe de carte et $xy \in E$. Si $xy$ n'est pas pincée, $G / xy$ est de carte
\end{prop}

\begin{proof}
    On raisonne par contraposée, supposons que $G / xy$ ne soit pas de carte. On se donne $H$ un témoin de $G$.
    $G / xy$ n'étant pas de carte, il n'admet pas de témoin. Si $xy$ est représentée dans $H$ par l'une des deux configurations
    de la figure \ref{IzEdge}, $G / xy$ admettrait un témoin ce qui n'est pas le cas. Notons $s$ le sommet faisant la liaison
    entre $x$ et $y$. On en déduit alors que $x, y$ ne sont pas dans la même face de $H$, donc $s$ est de degré au moins $4$,
    et il existe un cycle exploitant une arrête de la clique donnée par $s$, arrête que l'on notera $ua$, et contenant $x$
    en son intérieur sans perte de généralité
    \\~\\
    Si $x$ ne peut accéder à un sommet du cycle que par $s$, alors on peut placer $x$ et tout les sommets accessibles
    depuis ce dernier de manière à ce que $x$ et $y$ soient adjacents à une même face. On en déduit l'existence d'un deuxième
    cycle exploitant $s$ contenant $y$ en son intérieur, et de chemins reliant $x$ et $y$ à leurs cycles respectifs sans utiliser
    $s$. On peut même, pour simplifier, supposer que les deux cycles exploitent $ua$, en reliant les sommets adjacents à $s$ par
    une arrête subdivisée si ils sont adjacents à la même face, on obtient un témoin de $G$ dans lequel on peut agrandir
    l'un des cycles afin d'utiliser $ua$.
    \\~\\
    On note $C_x$ le cycle bordant $x$ et $C_y$ celui bordant $y$. On note $C_{x, y}$ la partie commune de ces deux cycles.
    On choisit $C_x$ et $C_y$ de manière à minimiser le nombre de sommets de $C_{x,y}$ d'abord, puis ceux de $C_x$ et $C_y$.
    Cela implique, en particulier, qu'il n'existe pas de chemin disjoint de $C_x$ d'un sommet de $C_{x,y}$ à $C_x$, car
    cela contredirait la minimalité (cela vaut pour $y$ également). On en déduit que dans les chemins de $x$ à $C_x$ dont
    l'existence a été montrée, au moins un a pour destination un sommet hors de $C_{x,y}$, et de même pour $y$.
    Le cas contraire, on pourrait construire un témoin comme illustré figure ??.
    \\~\\
    On suppose dans un premier temps que $C_{x,y}$ contient strictement $\{u, a\}$. On suppose même sans perte de généralité
    que $a$ a un voisin distinct de $u$ dans $C_{x, y}$. On se donne $z$ le sommet d'arrivée du chemin de $x$ à
    $C_x - C_{x,y}$ et $t$ celui de $y$.
    \\~\\
    Une disjonction de cas interminable termine la preuve (à bien faire sur papier)
\end{proof}

On dira qu'une suite de contractions est sans pincement si elle n'exploite aucune arrête pincée

\begin{cor}
    Si $G$ a un sous graphe induit se contractant sans pincement en un $K(3, H)$, $G$ n'est pas de carte
\end{cor}

\section{Caractérisation combinatoire des graphes de carte}

Ce qu'on essaie de montrer

\begin{conj}
    Tout graphe $G$ n'ayant pas de sous graphe induit se contractant sans pincement en un $K(3, H)$ est de carte
\end{conj}

\begin{lem}
    Un graphe $G$ se contracte en un $K(3, H)$ si et seulement si il existe $3$ sommets
    indépendants dans $G$ pouvant tous accéder aux $3$ même sommets par des chemins n'exploitant pas ces derniers (sauf en
    leur éxtrémités), ces derniers ne s'intersectant éventuellement qu'en leur fin (on situe leur début en les sommets indépendants)
\end{lem}

\begin{proof}
    Si $G$ vérifie cette propriété, alors en contractant les arrêtes en partant de la fin des chemins, on obtient un $K(3, H)$ : en
    effet les chemins étant distincts, à cause de leur première arrête car les sommets indépendants sont bien distincts,
    cette contraction transforme les $3$ chemins de chaque sommet indépendant en $3$ arrêtes (car les chemins n'exploitent pas les sommets
    d'arrivée), on a bien ce que l'on veut
    \\~\\
    Si $G$ se contracte en un $K(3, H)$, notons $V_1$, $V_2$, $V_3$ l'ensemble des sommets de $G$ formant l'indépendant
    de taille $3$ de $K(3, H)$, et $A_1, A_2, A_3$ ceux formant le graphe $H$. Soient $v_1, v_2, v_3$ dans chacun des ensembles respectifs.
    Ces $3$ sommets sont indépendants. Soient $a_1, a_2, a_3$ dans leurs ensembles respectifs. Comme il y a une arrête de $V_i$ à $A_j$
    pour tout $1 \leq i,j \leq 3$, il existe une arrête d'un élement de $V_i$ à un de $A_j$. $V_i$ étant obtenu par contraction d'arrêtes,
    $G[V_i]$ est connexe, tout comme $G[A_j]$. Ainsi, on trouve bien un chemin de $v_i$ à $a_j$. Un chemin de $v_k$ à $a_j$, $k \neq i$ construit
    de la même manière n'intersecte l'autre chemin qu'éventuellement dans la partie dans $A_j$, soit la fin du chemin.
\end{proof}

\begin{lem}\label{cycleCompl}
    Soit $G$ un graphe de carte à carte complète et $v \in V$ de degré au moins $2$. Pour un certain graphe témoin $H$, il existe un cycle séparant $v$
    du reste du graphe $H$ (le seul sommet dans l'intérieur du cycle est $v$), et tel que $v$ soit adjacent (dans $G$) à tout les sommets
    de $G$ formant ce cycle
\end{lem}

\begin{proof}
    On se donne une carte complète de $G$ et on confondra alors sommets et régions. Les voisins de $v$ sont exactement les régions
    rencontrant la frontière de $v$. Comme $v$ est homéomorphe à un disque, on se donne $\gamma : [0, 1] \rightarrow \mathbb{S}^2$
    une courbe de Jordan paramétrant sa frontière. Notons $u_0$ une région distincte de $v$ contenant $\gamma([t_0, \varepsilon[)$ pour $\varepsilon$
    assez petit, où $t_0 = 0$, cette dernière existe comme la carte est complète. On note $I_0 = \{ t \in [0, 1] \mid \gamma(t) \in u_0 \}$
    et $t_1 = \max(I_0)$. Si $t_1 = 1$, comme $v$ est de degré au moins $2$, il existe une autre région adjacente à $v$, en un point uniquement
    comme $t_1 = 1$. On peut alors montrer que $\gamma$ n'est pas un lacet contractile dans $u_0$ et donc que $u_0$ n'est pas simplement connexe, absurde.
    Donc $t_1 < 1$. Par maximalité, $\gamma(t_1)$ est sur le bord de $u_0$ et donc il existe un $u_1$ vérifiant les mêmes propriétés que $u_0$
    relativement à $t_1$. On a alors $u_0u_1 \in E$. On répète alors le processus afin de trouver $u_0, ..., u_k$, $k \geq 1$ formant un cycle
    \\~\\
    On construit alors le graphe témoin $H$ comme suit : on commence par se donner un graphe témoin quelconque de $G - v$ construit à partir de la carte
    complète donnée, que l'on considèrera alors plongé dans le plan. On ajoute ensuite $v$ : on ajoute au graphe témoin les sommets
    $\gamma(t_i)$, $1 \leq i \leq k$, à l'exception peut être de $\gamma(t_k)$ si $u_k = u_0$. On relie alors les regions $u_i$ aux $\gamma(t_j)$
    adjacents à ces dernières par des chemins, et on ajoute enfin un dernier point dans $v$, que l'on relie à chaque $\gamma(t_i)$.
    Par construction de $\gamma$, s'il existe une région adjacente à $v$ distincte des $u_i$, leur intersection est incluse
    dans les $\gamma(t_i)$, et il suffit alors de relier $\gamma(t_i)$ au point représentant la région correspondante pour ainsi avoir
    un témoin de $G$. Notons que $u_0, \gamma(t_1), u_1, \gamma(t_2), ..., u_0$ est un cycle séparant $v$ du reste de $H$ par construction de ce dernier,
    et la construction des $u_i$ implique que $v$ leur est adjacent dans $G$
\end{proof}

\begin{lem}\label{3connCompl}
    Soit $G$ un graphe de carte $3$-connexe. Il existe un surgraphe de $G$ à un sommet de plus admettant une carte complète.
\end{lem}

\begin{proof}
    On se donne un témoin $H$ de $G$ et on note $U$ l'ensemble des sommets autres que $V$ de $H$. On peut supposer que $H$
    est construit de telle sorte à ce que $d(u) \geq 2$ pour $u \in U$. Soit $u \in U$, $x, y \in V$ parmi ses voisins.
    Si $x$ et $y$ sont adjacents à la même face de $H$, alors on ajoute une arrête entre $x$ et $y$ à l'intérieur de cette face,
    que l'on subdivise ensuite à l'aide d'un sommet afin de garder le graphe biparti. On répète alors
    ce processus jusqu'à ce que tout les sommets adjancents dans $G$ soient reliés par deux chemins de longueur $2$.
    \\~\\
    Le nouveau graphe $H'$ est planaire biparti et $2$-connexe : si l'on retire un sommet de $V$ le graphe reste clairement
    connexe comme $G$ est $3$-connexe. Si l'on retire un sommet de $U'$ (l'ensemble $U$ avec les nouveaux sommets ajoutés),
    $H$ reste aussi connexe :
    si le sommet retiré est parmi ceux de $U' \backslash U$, la construction de $U'$ donne que le graphe reste connexe. Sinon,
    on se donne $v_1, ..., v_k$ les voisins de $u$ le sommet retiré, tels que $v_i$ et $v_{i+1}$ soient adjacents à la
    même face. On a alors qu'après le retrait de $u$, par construction encore de $U'$, les $v_i$ sont tous accessibles
    entre eux. On peut alors voir que cela implique que le graphe reste connexe et est un témoin de $G$
    \\~\\
    On construit alors un surgraphe de $H'$ que l'on va plonger dans le plan. On construit le
    surgraphe $H''$ en itérant sur tout sommet $v \in V$, et en ajoutant une arrête entre chaque paire $u_1, u_2$
    voisine de $v$ adjacente à une même face, arrête prenant la forme d'un chemin dans cette face. Si $v$ est
    un sommet extérieur dans $H'$, et que l'on considère $2$ de ses voisins dans $H'$ eux aussi extérieurs,
    alors on choisira le chemin de telle sorte à ce que $v$ devienne intérieur dans $H''$.
    Ainsi tout les élements de $U'$ deviennent de degré au moins $3$. $G$ étant $3$-connexe, on peut alors voir que $H''$
    le devient également (on ne peut plus isoler de sommet de $U'$). $H''$ est également planaire et par construction,
    tout les sommets $v \in V$ sont intérieurs.
    \\~\\
    \textbf{(merde faut montrer que les adjacences restent cohérentes aussi).}
    On construit à présent la carte : pour $v \in V$, la région associée à $v$ est l'adhérence de la face dans laquelle se trouve
    le point $v$ dans $H'' - v$. Cette face est bien homéomorphe à un disque, comme $H'' - v$ est $2$-connexe et qu'alors
    cette dernière est bordée par un cycle (voir ...), le théorème de Jordan-Schönflies nous donne ce que l'on veut.
    Cet ensemble de région correspond alors à l'intérieur du cycle de $H''$
    séparant la face non bornée des autres (comme $H''$ est $2$-connexe). Ainsi, pour les mêmes raisons, la face non bornée, vue dans $\mathbb{S}^2$,
    est homéomorphe à un disque. On ajoute alors un sommet à $G$ correspondant à cette face et à ses adjacences dans cette carte.
    La carte est bien complète par construction

\end{proof}

\begin{lem}
    Soit $G$ un graphe de carte et $v \in V$. Il existe un surgraphe de carte $G'$ de $G$ se contractant sans pincement en $G$
    tel que pour un certain témoin $H$ de $G'$, les voisins de $v$ dans $H$ soient tous de degré $2$
\end{lem}

\begin{proof}
    On se donne un témoin de $G$. On subdivise $2$ fois chaque arrête entre $v$ et un de ses voisins. Le graphe obtenu est bien
    planaire biparti et est donc le témoin d'un graphe de carte, qui est un surgraphe de $G$. On peut contracter $G'$
    en $G$ en contractant les arrêtes entre $v$ et ses voisins dans $G'$. Ces arrêtes ne sont pas pincées car le voisinage de $v$
    et d'un de ses voisins dans $G'$ quelconque est disjoint.
\end{proof}

\begin{lem}
    Soit $G$ un graphe ne possédant pas de sous graphe induit se contractant sans pincement en $K(3, H)$. Il existe un
    graphe $G'$ $3$-connexe tel que $G$ en soit sous graphe induit, vérifiant la même propriété
\end{lem}

\begin{proof}
    Deux idées de preuve : la première rajouter un seul sommet et le relier le plus possible jusqu'à ne plus pouvoir, montrer
    qu'alors on est $3$-connexe.\\
    La deuxième, rajouter des sommets entre chaque composante $3$-connexe et prier pour que ça garde la propriété
    de $K(3, H)$
\end{proof}

\section{Algorithmes}

On a déjà $2$ critères permettant de partiellement décider si un graphe quelconque est de carte : si il ne possède pas
de sous graphe se contractant en $K(3, H)$, le graphe est de carte. Si il possède un graphe se contractant sans pincement
(la propriété de pincement peut se vérifier en temps polynômial) en $K(3, H)$, il n'est pas de carte.
Etant donné ces deux trucs, on peut peut être alors préciser ce qu'il se trouve entre les deux, peut être par d'autres
résultats mathématiques ou alors en restreignant à des sous classes

\end{flushleft}
\end{document}