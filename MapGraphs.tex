\documentclass{scrartcl}
\usepackage[a4paper, total={7in, 10in}]{geometry}
\usepackage{stmaryrd}
\usepackage{amsthm}
\usepackage{amssymb}
\usepackage{amsmath}
\usepackage{algorithm2e}
\usepackage{hyperref}
\usepackage[french]{babel}
\usepackage{color}
\usepackage{tikz}
\usetikzlibrary{automata, arrows.meta, positioning}
\usepackage{MacrosArt}


\begin{document}

\title{Stage Map Graphs}

\author{José Lorgeré}

\maketitle

\begin{flushleft}

\section*{Introduction}

\section{Résultats}

\begin{theorem}[Caractérisation des graphes de carte]\label{carCarte}
    Un graphe $G = (V, E)$ est de carte si et seulement si il existe $H$ un graphe biparti planaire dont un des
    côtés de la bipartition est $V$, tel que $H^2[V] = G$.
\end{theorem}

\begin{prop}
    Si $G$ est un graphe de carte, et $H = G[A]$ où $A \subset V$, alors $H$ est un graphe de carte
\end{prop}

\begin{proof}
    On reprend la carte de $G$ où l'on ne garde que les régions identifées aux sommets présents dans $A$
\end{proof}

\begin{def*}[Join]
    Le join de $2$ graphes $G = (V, E)$ et $G' = (V', E')$ est le graphe ayant pour sommet $V'' = V \cup V'$ et pour arrêtes
    \[ E'' = E \cup E' \cup \{ xx' \mid x \in V, x \in V' \} \]
    On notera pour tout $n \geq 1$ $K(n, G)$ le join d'un indépendant à $n$ sommets avec le graphe $G$
\end{def*}

\begin{prop}
    Soit $G$ un graphe à $3$ sommets. $K(3, G)$ n'est pas un graphe de carte
\end{prop}

\begin{proof}
    Supposons par l'absurde que $K(3, G)$ est un graphe de carte. On se donne alors $H$ vérifiant toutes les propriétés
    du théorème \ref{carCarte}. On note pour tout $v_i$, sommet indépendant de $K(3, G)$, $N_i$ son voisinage dans $H$
    incluant $v_i$. On note $H'$ le graphe $(((H / N_1) / N_2) / N_3)$, qui est donc un mineur de $H$ et donc planaire.
    Or $H' = K_{3,3}$ : en effet $N_i$ est adjacent à tout les sommets de $G$, comme le voisinage de $v_i$ dans $H$ l'est,
    car $v_i$ l'est dans $K(3, G)$. De plus il n'y a pas d'arrêtes entre les sommets de $G$ comme $H$ est biparti, et pas
    d'arrêtes entre les différents $N_i$ comme les $v_i$ sont indépendants dans $G$.\\
    Absurde, $K(3, G)$ n'est pas un graphe de carte
\end{proof}

Un argument très similaire montre que les expansions de $K(3, G)$ ne sont également pas des graphes de carte.\\
Notons que $K(2, G)$ pour $G$ quelconque et $K(n, G)$ pour $G$ à $2$ sommets sont de carte (ptit dessin)

\begin{conj}
    Les graphes de carte sont stables par contraction d'arrête
\end{conj}

\begin{conj}
    Si $G$ n'a pas de sous graphe induit se contractant en un $K(3, H)$, $G$ est de carte
\end{conj}

Si la conjecture est vraie, on a alors un algorithme polynômial pour reconnaître les cographes de carte : pour l'union disjointe,
il n'y a rien à vérifier. Pour le join, il suffit de vérifier récursivement que chaque composante a join est de carte, et
qu'aucune ne contient un indépendant de taille $3$. Au final on devrait avoir du $O(n^4)$ pire des cas avec que des algos
naïfs, mais y a sûrement moyen d'améliorer

\end{flushleft}
\end{document}