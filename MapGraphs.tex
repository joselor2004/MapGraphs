\documentclass{scrartcl}
\usepackage[a4paper, total={7in, 10in}]{geometry}
\usepackage{stmaryrd}
\usepackage{amsthm}
\usepackage{amssymb}
\usepackage{amsmath}
\usepackage{algorithm2e}
\usepackage{hyperref}
\usepackage[french]{babel}
\usepackage{color}
\usepackage{tikz}
\usetikzlibrary{automata, arrows.meta, positioning}
\usepackage{MacrosArt}


\begin{document}

\title{Stage Map Graphs}

\author{José Lorgeré}

\maketitle

\begin{flushleft}

\section*{Introduction}

\section{Résultats}

\subsection{Trucs généraux}

\begin{def*}[Carte]
    Une carte est une fonction $f : V \rightarrow \mathcal{P}(\mathbb{S}^2)$ telle que pour tout $v \in V$, $f(v)$
    est homéomorphe à $\mathbb{D}^2$ et telle que pour $v \neq u$, $f(u)$ et $f(v)$ sont d'intérieur disjoints. Si
    $f$ forme un recouvrement de $\mathbb{S}^2$, on la dira sans trou, ou complète. On appelera les $f(v)$ régions
\end{def*}

\begin{def*}[Graphe de carte]
    Un graphe $G = (V, E)$ est de carte s'il existe une carte $f$ sur $V$ telle que $xy \in E$ si et seulement si
    $f(x) \cap f(y) \neq \varnothing$
\end{def*}

\begin{theorem}[Caractérisation des graphes de carte]\label{carCarte}
    Un graphe $G = (V, E)$ est de carte si et seulement si il existe $H$ un graphe biparti planaire dont un des
    côtés de la bipartition est $V$, tel que $H^2[V] = G$. Un tel graphe $H$ est appelé graphe témoin de $G$
\end{theorem}

\begin{prop}
    Si $G$ est un graphe de carte, et $H = G[A]$ où $A \subset V$, alors $H$ est un graphe de carte
\end{prop}

\begin{proof}
    On reprend la carte de $G$ où l'on ne garde que les régions identifées aux sommets présents dans $A$
\end{proof}

\begin{def*}[Join]
    Le join de $2$ graphes $G = (V, E)$ et $G' = (V', E')$ est le graphe ayant pour sommet $V'' = V \cup V'$ et pour arrêtes
    \[ E'' = E \cup E' \cup \{ xx' \mid x \in V, x \in V' \} \]
    On notera pour tout $n \geq 1$ $K(n, G)$ le join d'un indépendant à $n$ sommets avec le graphe $G$
\end{def*}

\begin{prop}
    Soit $G$ un graphe à $3$ sommets. $K(3, G)$ n'est pas un graphe de carte
\end{prop}

\begin{proof}
    Supposons par l'absurde que $K(3, G)$ est un graphe de carte. On se donne alors $H$ vérifiant toutes les propriétés
    du théorème \ref{carCarte}. On note pour tout $v_i$, sommet indépendant de $K(3, G)$, $N_i$ son voisinage dans $H$
    incluant $v_i$. On note $H'$ le graphe $(((H / N_1) / N_2) / N_3)$, qui est donc un mineur de $H$ et donc planaire.
    Or $H' = K_{3,3}$ : en effet $N_i$ est adjacent à tout les sommets de $G$, comme le voisinage de $v_i$ dans $H$ l'est,
    car $v_i$ l'est dans $K(3, G)$. De plus il n'y a pas d'arrêtes entre les sommets de $G$ comme $H$ est biparti, et pas
    d'arrêtes entre les différents $N_i$ comme les $v_i$ sont indépendants dans $G$.\\
    Absurde, $K(3, G)$ n'est pas un graphe de carte
\end{proof}

Un argument très similaire montre que les graphes se contractant en $K(3, G)$ ne sont également pas des graphes de carte.\\
Notons que $K(2, G)$ pour $G$ quelconque et $K(n, G)$ pour $G$ à $2$ sommets sont de carte, ainsi que le join de $G$ et $H$ où
$G$ et $H$ ont $3$ sommets et possèdent tout deux des arrêtes (ptit dessin)
\\~\\
Pour la suite dès que l'on utilisera la notation $K(3, H)$, on supposera $H$ à $3$ sommets
\begin{conj}
    Si $G$ n'a pas de sous graphe induit se contractant en un $K(3, H)$, $G$ est de carte
\end{conj}

Si la conjecture est vraie, on a alors un algorithme polynômial pour reconnaître les cographes de carte : pour l'union disjointe,
il n'y a rien à vérifier. Pour le join, il suffit de vérifier récursivement que chaque composante a join est de carte, et
qu'aucune ne contient un indépendant de taille $3$. Au final on devrait avoir du $O(n^4)$ pire des cas avec que des algos
naïfs, mais y a sûrement moyen d'améliorer

\subsection{Preuve de la conjecture}

On va regarder des graphes dont les cartes sont complètes et $2$-connexes. (d'ailleurs y a moyen que carte complète
$\Rightarrow$ $2$-connexe)
On aimerait pouvoir trouver des classes spéciales de graphes de carte, dans l'idéal définies de manière combinatoire et
relativement stables, qui admettent des cartes complètes

\begin{conj}
    Il existe une classe de graphe $\mathcal{C}$ telle que si $G \in \mathcal{C}$ et est de carte, $G$ admet une carte complète.
    De plus cette classe $\mathcal{C}$ est stable par contraction d'une certaine arrête
\end{conj}

\begin{conj}
    Il s'agit de l'ensemble des graphes $3$-connexes
\end{conj}

\subsubsection{Bases combinatoire}

\begin{lem}
    Un graphe $G$ se contracte en un $K(3, H)$ si et seulement si il existe $3$ sommets
    indépendants dans $G$ pouvant tous accéder aux $3$ même sommets par des chemins n'exploitant pas ces derniers (sauf en
    leur éxtrémités), ces derniers ne s'intersectant éventuellement qu'en leur fin (on situe leur début en les sommets indépendants)
\end{lem}

\begin{proof}
    Si $G$ vérifie cette propriété, alors en contractant les arrêtes en partant de la fin des chemins, on obtient un $K(3, H)$ : en
    effet les chemins étant distincts, à cause de leur première arrête car les sommets indépendants sont bien distincts,
    cette contraction transforme les $3$ chemins de chaque sommet indépendant en $3$ arrêtes (car les chemins n'exploitent pas les sommets
    d'arrivée), on a bien ce que l'on veut
    \\~\\
    Si $G$ se contracte en un $K(3, H)$, notons $V_1$, $V_2$, $V_3$ l'ensemble des sommets de $G$ formant l'indépendant
    de taille $3$ de $K(3, H)$, et $A_1, A_2, A_3$ ceux formant le graphe $H$. Soient $v_1, v_2, v_3$ dans chacun des ensembles respectifs.
    Ces $3$ sommets sont indépendants. Soient $a_1, a_2, a_3$ dans leurs ensembles respectifs. Comme il y a une arrête de $V_i$ à $A_j$
    pour tout $1 \leq i,j \leq 3$, il existe une arrête d'un élement de $V_i$ à un de $A_j$. $V_i$ étant obtenu par contraction d'arrêtes,
    $G[V_i]$ est connexe, tout comme $G[A_j]$. Ainsi, on trouve bien un chemin de $v_i$ à $a_j$. Un chemin de $v_k$ à $a_j$, $k \neq i$ construit
    de la même manière n'intersecte l'autre chemin qu'éventuellement dans la partie dans $A_j$, soit la fin du chemin.
\end{proof}

\subsubsection{Etude des cartes}

\begin{lem}\label{cycleCompl}
    Soit $G$ un graphe de carte à carte complète et $v \in V$ de degré au moins $2$. Pour un certain graphe témoin $H$, il existe un cycle séparant $v$
    du reste du graphe $H$ (le seul sommet dans l'intérieur du cycle est $v$), et tel que $v$ soit adjacent (dans $G$) à tout les sommets
    de $G$ formant ce cycle
\end{lem}

\begin{proof}
    On se donne une carte complète de $G$ et on confondra alors sommets et régions. Les voisins de $v$ sont exactement les régions
    rencontrant la frontière de $v$. Comme $v$ est homéomorphe à un disque, on se donne $\gamma : [0, 1] \rightarrow \mathbb{S}^2$
    une courbe de Jordan paramétrant sa frontière. Notons $u_0$ une région distincte de $v$ contenant $\gamma([t_0, \varepsilon[)$ pour $\varepsilon$
    assez petit, où $t_0 = 0$, cette dernière existe comme la carte est complète. On note $I_0 = \{ t \in [0, 1] \mid \gamma(t) \in u_0 \}$
    et $t_1 = \max(I_0)$. Si $t_1 = 1$, comme $v$ est de degré au moins $2$, il existe une autre région adjacente à $v$, en un point uniquement
    comme $t_1 = 1$. On peut alors montrer que $\gamma$ n'est pas un lacet contractile dans $u_0$ et donc que $u_0$ n'est pas simplement connexe, absurde.
    Donc $t_1 < 1$. Par maximalité, $\gamma(t_1)$ est sur le bord de $u_0$ et donc il existe un $u_1$ vérifiant les mêmes propriétés que $u_0$
    relativement à $t_1$. On a alors $u_0u_1 \in E$. On répète alors le processus afin de trouver $u_0, ..., u_k$, $k \geq 1$ formant un cycle
    \\~\\
    On construit alors le graphe témoin $H$ comme suit : on commence par se donner un graphe témoin quelconque de $G - v$ construit à partir de la carte
    complète donnée, que l'on considèrera alors plongé dans le plan. On ajoute ensuite $v$ : on ajoute au graphe témoin les sommets
    $\gamma(t_i)$, $1 \leq i \leq k$, à l'exception peut être de $\gamma(t_k)$ si $u_k = u_0$. On relie alors les regions $u_i$ aux $\gamma(t_j)$
    adjacents à ces dernières par des chemins, et on ajoute enfin un dernier point dans $v$, que l'on relie à chaque $\gamma(t_i)$.
    Par construction de $\gamma$, s'il existe une région adjacente à $v$ distincte des $u_i$, leur intersection est incluse
    dans les $\gamma(t_i)$, et il suffit alors de relier $\gamma(t_i)$ au point représentant la région correspondante pour ainsi avoir
    un témoin de $G$. Notons que $u_0, \gamma(t_1), u_1, \gamma(t_2), ..., u_0$ est un cycle séparant $v$ du reste de $H$ par construction de ce dernier,
    et la construction des $u_i$ implique que $v$ leur est adjacent dans $G$
\end{proof}

\begin{lem}\label{3connCompl}
    Soit $G$ un graphe de carte $3$-connexe. Il existe un surgraphe de $G$ à un sommet de plus admettant une carte complète.
\end{lem}

\begin{proof}
    On se donne un témoin $H$ de $G$ et on note $U$ l'ensemble des sommets autres que $V$ de $H$. On peut supposer que $H$
    est construit de telle sorte à ce que $d(u) \geq 2$ pour $u \in U$. Soit $u \in U$, $x, y \in V$ parmi ses voisins.
    Si $x$ et $y$ sont adjacents à la même face de $H$, alors on ajoute une arrête entre $x$ et $y$ à l'intérieur de cette face,
    que l'on subdivise ensuite à l'aide d'un sommet afin de garder le graphe biparti. On répète alors
    ce processus jusqu'à ce que tout les sommets adjancents dans $G$ soient reliés par deux chemins de longueur $2$.
    \\~\\
    Le nouveau graphe $H'$ est planaire biparti et $2$-connexe : si l'on retire un sommet de $V$ le graphe reste clairement
    connexe comme $G$ est $3$-connexe. Si l'on retire un sommet de $U'$ (l'ensemble $U$ avec les nouveaux sommets ajoutés),
    $H$ reste aussi connexe :
    si le sommet retiré est parmi ceux de $U' \backslash U$, la construction de $U'$ donne que le graphe reste connexe. Sinon,
    on se donne $v_1, ..., v_k$ les voisins de $u$ le sommet retiré, tels que $v_i$ et $v_{i+1}$ soient adjacents à la
    même face. On a alors qu'après le retrait de $u$, par construction encore de $U'$, les $v_i$ sont tous accessibles
    entre eux. On peut alors voir que cela implique que le graphe reste connexe et est un témoin de $G$
    \\~\\
    On construit alors un surgraphe de $H'$ que l'on va plonger dans le plan. On construit le
    surgraphe $H''$ en itérant sur tout sommet $v \in V$, et en ajoutant une arrête entre chaque paire $u_1, u_2$
    voisine de $v$ adjacente à une même face, arrête prenant la forme d'un chemin dans cette face. Si $v$ est
    un sommet extérieur dans $H'$, et que l'on considère $2$ de ses voisins dans $H'$ eux aussi extérieurs,
    alors on choisira le chemin de telle sorte à ce que $v$ devienne intérieur dans $H''$.
    Ainsi tout les élements de $U'$ deviennent de degré au moins $3$. $G$ étant $3$-connexe, on peut alors voir que $H''$
    le devient également (on ne peut plus isoler de sommet de $U'$). $H''$ est également planaire et par construction,
    tout les sommets $v \in V$ sont intérieurs.
    \\~\\
    (merde faut montrer que les adjacences restent cohérentes aussi).
    On construit à présent la carte : pour $v \in V$, la région associée à $v$ est l'adhérence de la face dans laquelle se trouve
    le point $v$ dans $H'' - v$. Cette face est bien homéomorphe à un disque, comme $H'' - v$ est $2$-connexe et qu'alors
    cette dernière est bordée par un cycle (voir ...), le théorème de Jordan-Schönflies nous donne ce que l'on veut.
    Cet ensemble de région correspond alors à l'intérieur du cycle de $H''$
    séparant la face non bornée des autres (comme $H''$ est $2$-connexe). Ainsi, pour les mêmes raisons, la face non bornée, vue dans $\mathbb{S}^2$,
    est homéomorphe à un disque. On ajoute alors un sommet à $G$ correspondant à cette face et à ses adjacences dans cette carte.
    La carte est bien complète par construction

\end{proof}

\begin{lem}
    Soit $G$ un graphe $3$-connexe ne contenant pas un sous graphe induit sur contractant en $K(3, H)$. Alors $G$ est de carte
\end{lem}

\begin{proof}
    On procède par récurrence sur le nombre de sommets de $G$ que l'on note $n$.
    Tout les graphes a moins de $n = 4$ sommets sont de carte, on a notre initialisation.\\
    Soit $G$ un graphe a $n \geq 5$ sommets vérifiant l'hypothèse. On se donne par ..., une arrête $xy \in E$ telle que $G / xy$ soit $3$-connexe.
    $G / xy$ ne possède pas de sous graphe se contractant en $K(3, H)$, sinon $G$ en possède un, $G / xy$ est alors de carte par récurrence. On ajoute
    un sommet $\infty$ à $G / xy$ que l'on rend adjacent à certains sommets de telle sorte que $G'$, le nouveau graphe, admette une carte complète par
    le lemme \ref{3connCompl}. Notons $z$ le sommet obtenu par contraction de $xy$. $G / xy$ étant $3$-connexe, $z$ est de degré au moins $3$ dans $G'$.
    \\~\\
    On traite d'abord le cas où $z$ n'est pas adjacent à $\infty$. On se donne $H$ un témoin de $G'$ vérifiant les hypothèse du lemme \ref{3connCompl}
    relativement à $z$. On note $x_1, ..., x_k$ les voisins de $x$ dans $G'$ présents dans le cycle, l'ordre donné étant l'ordre de leur apparition
    dans le cycle. On note pour tout $i$ $P_i$ le chemin de $x_i$ à $x_{i+1}$ dans $H$ donné par le sens usuel du cycle (on considèrera tout les indices
    modulo $k$). On va à présent construire un témoin pour $G$. On distingue ensuite plusieurs cas selon la position des voisins de $y$ dans $H$ :
    \begin{itemize}
        \item Si tout les voisins de $y$ sont compris dans un $P_i$, on place $y$ dans l'intérieur du cycle donné par $x$ et $P_i$, et on
        relie alors $y$ à ses voisins par des chemins dans l'intérieur du cycle. On ajoute également un chemin de $y$ vers $x$ que l'on
        subdivise pour garder le graphe biparti

        \item 
    \end{itemize}
\end{proof}

Schéma de preuve : se donner un graphe qui marche pour avoir des cartes complètes, de telle sorte que la contraction d'arrête
garde cette propriété. On contracte la bonne arrête. On retire le sommet identifé et on identifie un cycle minimal qui
l'entoure dans le $H$ (on est de taille 1 de moins donc de carte). Puis, on distingue globalement $3$ cas dont un
problématique, qui sont explicités dans le livre. Le cas $K_5$ ne pose pas de problème car on a droit à pleins
d'intersections entre régions, le cas $K_{3,3}$ qui devient $K(3, G)$ est évacué par l'hypothèse initiale
\\~\\
Pour conclure, il faut également dire que l'on peut ajouter des nouveaux sommets à un graphe sans $K(3, G)$
pour obtenir un graphe vérifiant les hypothèses et toujours sans $K(3, G)$. La stabilité par sous graphe induit finit
la preuve
\end{flushleft}
\end{document}