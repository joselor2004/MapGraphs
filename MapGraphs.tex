\documentclass{scrartcl}
\usepackage[a4paper, total={7in, 10in}]{geometry}
\usepackage{stmaryrd}
\usepackage{amsthm}
\usepackage{amssymb}
\usepackage{amsmath}
\usepackage{algorithm2e}
\usepackage{hyperref}
\usepackage{comment}
\usepackage[french]{babel}
\usepackage{color}
\usepackage{tikz}
\usetikzlibrary{automata, arrows.meta, positioning}
\usepackage{MacrosArt}


\begin{document}

\title{Stage Map Graphs}

\author{José Lorgeré}

\maketitle

\begin{flushleft}

\section*{Introduction}

\section{Trucs généraux}

\subsection{Résultats et définitions préliminaires}

\begin{def*}[Carte]
    Une carte est une fonction $f : V \rightarrow \mathcal{P}(\mathbb{S}^2)$ telle que pour tout $v \in V$, $f(v)$
    est homéomorphe à $\mathbb{D}^2$ et telle que pour $v \neq u$, $f(u)$ et $f(v)$ sont d'intérieur disjoints. Si
    $f$ forme un recouvrement de $\mathbb{S}^2$, on la dira sans trou, ou complète. On appelera les $f(v)$ régions
\end{def*}

\begin{def*}[Graphe de carte]
    Un graphe $G = (V, E)$ est de carte s'il existe une carte $f$ sur $V$ telle que $xy \in E$ si et seulement si
    $f(x) \cap f(y) \neq \varnothing$
\end{def*}

\begin{theorem}[Caractérisation des graphes de carte]\label{carCarte}
    Un graphe $G = (V, E)$ est de carte si et seulement si il existe $H$ un graphe biparti planaire dont un des
    côtés de la bipartition est $V$, tel que $H^2[V] = G$. Un tel graphe $H$ est appelé graphe témoin de $G$
\end{theorem}

\begin{prop}
    Si $G$ est un graphe de carte, et $H = G[A]$ où $A \subset V$, alors $H$ est un graphe de carte
\end{prop}

\begin{proof}
    On reprend la carte de $G$ où l'on ne garde que les régions identifées aux sommets présents dans $A$
\end{proof}

\begin{def*}[Join]
    Le join de $2$ graphes $G = (V, E)$ et $G' = (V', E')$ est le graphe ayant pour sommet $V'' = V \cup V'$ et pour arrêtes
    \[ E'' = E \cup E' \cup \{ xx' \mid x \in V, x \in V' \} \]
    On notera pour tout $n \geq 1$ $K(n, G)$ le join d'un indépendant à $n$ sommets avec le graphe $G$
\end{def*}

\begin{prop}
    Soit $G$ un graphe à $3$ sommets. $K(3, G)$ n'est pas un graphe de carte
\end{prop}

\begin{proof}
    Supposons par l'absurde que $K(3, G)$ est un graphe de carte. On se donne alors $H$ vérifiant toutes les propriétés
    du théorème \ref{carCarte}. On note pour tout $v_i$, sommet indépendant de $K(3, G)$, $N_i$ son voisinage dans $H$
    incluant $v_i$. On note $H'$ le graphe $(((H / N_1) / N_2) / N_3)$, qui est donc un mineur de $H$ et donc planaire.
    Or $H' = K_{3,3}$ : en effet $N_i$ est adjacent à tout les sommets de $G$, comme le voisinage de $v_i$ dans $H$ l'est,
    car $v_i$ l'est dans $K(3, G)$. De plus il n'y a pas d'arrêtes entre les sommets de $G$ comme $H$ est biparti, et pas
    d'arrêtes entre les différents $N_i$ comme les $v_i$ sont indépendants dans $G$.\\
    Absurde, $K(3, G)$ n'est pas un graphe de carte
\end{proof}

\begin{comment}

\subsection{Contractibilité des arrêtes}

On aimerait donner une condition nécessaire et suffisante permettant de contracter les arrêtes d'un graphe de carte.
Certaines arrêtes sont clairement contractibles, comme celles représentées par $2$ régions partageant une courbe, d'autres ne le sont
pas, comme l'arrête $xy$ du graphe $G_{pch}$ \ref{Gpch}, qui est un graphe de carte.
On dénotera par $\mathcal{G}_{pch}$ l'ensemble des graphes obtenu depuis $G_{pch}$,
en ajoutant des arrêtes entre les sommets distincts de $a, b, c$

\begin{figure}[h]
    \caption{Le graphe $G_{pch}$}\label{Gpch}
    \begin{center}
        \begin{tikzpicture}[auto]
            \begin{scope}[every node/.style={circle, draw}]
                \node (a) {$a$};
                \node (b) [right = 20mm of a] {$b$};
                \node (c) [right = of b] {$c$};
                \node (x) [below left = 12mm and 5mm of a] {$x$};
                \node (y) [below right = 12mm and 5mm of a] {$y$};
                \node (u) [below = of b] {$u$};
                \node (v) [below = of c] {$v$};

                \node (a1) [right = 20mm of v] {$a$};
                \node (b1) [right = of a1] {$b$};
                \node (c1) [right = of b1] {$c$};
                \node (d1) [right = of c1] {$d$};
                \node (x1) [above = of a1] {$x$};
                \node (y1) [above = of b1] {$y$};
                \node (u1) [above = of c1] {$u$};
                \node (v1) [above = of d1] {$v$};
            \end{scope}

            \path
            (a) edge (x) edge (y) edge (u) edge (v)
            (b) edge (x) edge (u) edge (v)
            (c) edge (y) edge (u) edge (v)
            (x) edge (y) edge[bend right] (u)
            (y) edge (u)
            (x1) edge (y1) edge (a1) edge (b1) edge (c1)
            (y1) edge (a1) edge (b1) edge (d1)
            (u1) edge (a1) edge (c1) edge (d1)
            (v1) edge (b1) edge (c1) edge (d1)
            (a1) edge (b1);
        \end{tikzpicture}

    \end{center}
\end{figure}

Il sera utile de savoir contracter certaines arrêtes particulières des graphes de carte. Si $G$ est un graphe de carte
et $H$ un témoin de $G$, si l'arrête $xy$ est représentée par une des $2$ configurations de la figure \ref{IzEdge}, alors
on peut la contracter en maintenant $G$ de carte : pour la première il s'agit de deux contractions d'arrêtes dans un graphe
planaire, pour la deuxième, il suffit d'ajouter une arrête subdivisée entre $x$ et $y$ passant par l'intérieur de la face à laquelle
ces deux derniers sont adjacents pour obtenir un témoin de $G$ présentant le premier cas

\begin{figure}[h]
    \caption{Cas d'arrêtes dont la contraction laisse $G$ de carte}\label{IzEdge}
    \begin{center}
        Le deuxième cas se généralise à des cliques arbitrairement grandes, dès que $x$ et $y$ sont adjacents à une même face
        \\~\\
        \begin{tikzpicture}[auto]
            \begin{scope}[every node/.style={circle, draw}]
                \node (x1) {$x$};
                \node (x2) [right = 50mm of x1] {$x$};
            \end{scope}

            \begin{scope}[every node/.style={rectangle, draw, fill=black}]
                \node (u) [right = of x1] {$\0$};
                \node (v) [right = of x2] {$\0$};
            \end{scope}

            \begin{scope}[every node/.style={circle, draw}]
                \node (y1) [right = of u] {$y$};
                \node (y2) [above = of v] {$y$};
                \node (z) [below = of v] {$z$};
                \node (t) [right = of v] {$t$};
            \end{scope}

            \path
            (u) edge (x1) edge (y1)
            (v) edge (x2) edge (y2) edge (z) edge (t);
        \end{tikzpicture}

    \end{center}
\end{figure}

\begin{def*}[Arrête pincée]
    On appellera arrête pincée de $G$ toute arrête $xy$ telle qu'il existe un sous graphe induit de $G$ contenant $xy$ se contractant en
    $G_p \in \mathcal{G}_{pch}$, et telle que la contraction n'exploite pas les arrêtes du sous graphe induit par $x,y,a,u$.
\end{def*}


(Ouais jpense pas continuer à parler des arrêtes pincées c trop dur)

\begin{prop}
    Soit $G$ un graphe de carte et $xy \in E$. Si $xy$ n'est pas pincée, $G / xy$ est de carte
\end{prop}

\begin{proof}
    On raisonne par contraposée, supposons que $G / xy$ ne soit pas de carte. On se donne $H$ un témoin de $G$.
    $G / xy$ n'étant pas de carte, il n'admet pas de témoin. Si $xy$ est représentée dans $H$ par l'une des deux configurations
    de la figure \ref{IzEdge}, $G / xy$ admettrait un témoin ce qui n'est pas le cas. Notons $s$ le sommet faisant la liaison
    entre $x$ et $y$. On en déduit alors que $x, y$ ne sont pas dans la même face de $H$, donc $s$ est de degré au moins $4$,
    et il existe un cycle exploitant une arrête de la clique donnée par $s$, arrête que l'on notera $ua$, et contenant $x$
    en son intérieur sans perte de généralité
    \\~\\
    Si $x$ ne peut accéder à un sommet du cycle que par $s$, alors on peut placer $x$ et tout les sommets accessibles
    depuis ce dernier de manière à ce que $x$ et $y$ soient adjacents à une même face. On en déduit l'existence d'un deuxième
    cycle exploitant $s$ contenant $y$ en son intérieur, et de chemins reliant $x$ et $y$ à leurs cycles respectifs sans utiliser
    $s$. On peut même, pour simplifier, supposer que les deux cycles exploitent $ua$, en reliant les sommets adjacents à $s$ par
    une arrête subdivisée si ils sont adjacents à la même face, on obtient un témoin de $G$ dans lequel on peut agrandir
    l'un des cycles afin d'utiliser $ua$.
    \\~\\
    On note $C_x$ le cycle bordant $x$ et $C_y$ celui bordant $y$. On note $C_{x, y}$ la partie commune de ces deux cycles.
    On choisit $C_x$ et $C_y$ de manière à minimiser le nombre de sommets de $C_{x,y}$ d'abord, puis ceux de $C_x$ et $C_y$.
    Cela implique, en particulier, qu'il n'existe pas de chemin disjoint de $C_x$ d'un sommet de $C_{x,y}$ à $C_x$, car
    cela contredirait la minimalité (cela vaut pour $y$ également). On en déduit que dans les chemins de $x$ à $C_x$ ne passant pas
    par $s$, dont l'existence a été montrée plus tôt, au moins un a pour destination un sommet hors de $C_{x,y}$, et de même pour $y$.
    Le cas contraire, on pourrait construire un témoin comme illustré figure ??.
    \\~\\
    On suppose dans un premier temps que $C_{x,y}$ est strictement plus grand que $\{u, s, a\}$. On suppose même, sans perte de généralité,
    que $a$ possède un voisin $q$ dans $C_{x,y}$ distinct de $s$. Cela implique qu'il existe un sommet $p \in V \cap C_{x,y}$, distinct de $u$
    et $a$, tel que $ap$ soit une arrête de $G$ : en effet si ce n'était pas le cas, les voisins de $q$ distincts de $a$
    dans $C_x$ ou $C_y$ ne seraient pas dans $C_{x, y}$. Soit $r$ un de ces voisins, supposons dans $C_x$.
    $r$ et $a$ sont adjacents à la même face : si ce n'est pas le cas, il existe alors un chemin de $q$ à un sommet du cycle contenant $r$
    séparant $r$ et $a$. On se donne alors $r'$ le voisin de $q$ sur ce chemin, et on réitère le processus jusqu'à obtenir un point $r^*$
    adjacent à la même face que $a$. Quitte à faire un bon choix de témoin initialement, on peut supposer que les sommets de $V$ formant
    une arrête et adjacents à une même face de $H$ sont reliés par une arrête subdivisée. Notons $q'$ le sommet entre $a$ et $r^*$.
    On peut alors construire un nouveau cycle contenant $x$ en exploitant les arrête $aq', q'r^*$ puis le chemin contenant $r^*$ et $C_x$.
    Ce cycle a une partie commune avec $C_y$ strictement plus petite que $C_{x,y}$, cela contredit la minimalité.
    \\~\\
    On distingue alors $2$ grands cas
    \begin{itemize}
        \item Si $zt$ n'est pas une arrête de $G$ : alors en considérant les sommets $a, z, t$ et $x, y, u, p$, on obtient par de bon
        choix de chemins des $a, z, t$ vers les $x, y, u, p$ un graphe se contractant de la manière voulue en un $G_p \in \mathcal{G}_{pch}$

        \item Si $zt$ est une arrête de $G$ : si il n'existe aucun autre chemin hors cycle de $C_x$ à $C_y$, le graphe $G / xy$
        est de carte, il en existe donc au moins un. 
    \end{itemize}
\end{proof}

On dira qu'une suite de contractions est sans pincement si elle n'exploite aucune arrête pincée

\begin{cor}
    Si $G$ a un sous graphe induit se contractant sans pincement en un $K(3, H)$, $G$ n'est pas de carte
\end{cor}

\end{comment}

\section{Etude des cartes complètes $3$-connexes}

On va faire qqes lemmes sur les cartes complètes qui permettront de déblayer un algo

\begin{lem}[Cycle recouvrant]\label{cycleCompl}
    Soit $G$ un graphe de carte $3$-connexe à carte complète. Il existe un cycle, ou chemin de taille $2$, $C$ inclus dans le voisinage de $v$,
    tel que $N(v) \subset N(C)$
\end{lem}

Faut prendre en compte le cas où la fusion de $v$ et d'une région n'est pas simplement connexe et intersecte en au moins $2$ segments
disjoints

\begin{proof}
    On se donne une carte complète de $G$ et on confondra alors sommets et régions. Les voisins de $v$ sont exactement les régions
    rencontrant la frontière de $v$. Comme $v$ est homéomorphe à un disque, on se donne $\gamma : [0, 1] \rightarrow \mathbb{S}^2$
    une courbe de Jordan paramétrant sa frontière. Notons $u_0$ une région distincte de $v$ contenant $\gamma([t_0, \varepsilon[)$ pour $\varepsilon$
    assez petit, où $t_0 = 0$, cette dernière existe comme la carte est complète. On note $I_0 = \{ t \in [0, 1] \mid \gamma(t) \in u_0 \}$
    et $t_1 = \max(I_0)$. Si $t_1 = 1$, comme $v$ est de degré au moins $2$, il existe une autre région adjacente à $v$, en un point uniquement
    comme $t_1 = 1$. On peut alors montrer que $\gamma$ n'est pas un lacet contractile dans $u_0$ et donc que $u_0$ n'est pas simplement connexe, absurde.
    Donc $t_1 < 1$. Par maximalité, $\gamma(t_1)$ est sur le bord de $u_0$ et donc il existe un $u_1$ vérifiant les mêmes propriétés que $u_0$
    relativement à $t_1$. On a alors $u_0u_1 \in E$. On répète alors le processus afin de trouver $u_0, ..., u_k$, $k \geq 1$ formant un cycle
    \\~\\
    On construit alors le graphe témoin $H$ comme suit : on commence par se donner un graphe témoin quelconque de $G - v$ construit à partir de la carte
    complète donnée, que l'on considèrera alors plongé dans le plan. On ajoute ensuite $v$ : on ajoute au graphe témoin les sommets
    $\gamma(t_i)$, $1 \leq i \leq k$, à l'exception peut être de $\gamma(t_k)$ si $u_k = u_0$. On relie alors les regions $u_i$ aux $\gamma(t_j)$
    adjacents à ces dernières par des chemins, et on ajoute enfin un dernier point dans $v$, que l'on relie à chaque $\gamma(t_i)$.
    Par construction de $\gamma$, s'il existe une région adjacente à $v$ distincte des $u_i$, leur intersection est incluse
    dans les $\gamma(t_i)$, et il suffit alors de relier $\gamma(t_i)$ au point représentant la région correspondante pour ainsi avoir
    un témoin de $G$. Notons que $u_0, \gamma(t_1), u_1, \gamma(t_2), ..., u_0$ est un cycle séparant $v$ du reste de $H$ par construction de ce dernier,
    et la construction des $u_i$ implique que $v$ leur est adjacent dans $G$
\end{proof}

\begin{lem}\label{3connCompl}
    Soit $G$ un graphe de carte $3$-connexe. Il existe un surgraphe de $G$ à un sommet de plus admettant une carte complète.
\end{lem}

\begin{proof}
    On se donne un témoin $H$ de $G$ et on note $U$ l'ensemble des sommets autres que $V$ de $H$. On peut supposer que $H$
    est construit de telle sorte à ce que $d(u) \geq 2$ pour $u \in U$. Soit $u \in U$, $x, y \in V$ parmi ses voisins.
    Si $x$ et $y$ sont adjacents à la même face de $H$, alors on ajoute une arrête entre $x$ et $y$ à l'intérieur de cette face,
    que l'on subdivise ensuite à l'aide d'un sommet afin de garder le graphe biparti. On répète alors
    ce processus jusqu'à ce que tout les sommets adjancents dans $G$ soient reliés par deux chemins de longueur $2$.
    \\~\\
    Le nouveau graphe $H'$ est planaire biparti et $2$-connexe : si l'on retire un sommet de $V$ le graphe reste clairement
    connexe comme $G$ est $3$-connexe. Si l'on retire un sommet de $U'$ (l'ensemble $U$ avec les nouveaux sommets ajoutés),
    $H$ reste aussi connexe :
    si le sommet retiré est parmi ceux de $U' \backslash U$, la construction de $U'$ donne que le graphe reste connexe. Sinon,
    on se donne $v_1, ..., v_k$ les voisins de $u$ le sommet retiré, tels que $v_i$ et $v_{i+1}$ soient adjacents à la
    même face. On a alors qu'après le retrait de $u$, par construction encore de $U'$, les $v_i$ sont tous accessibles
    entre eux. On peut alors voir que cela implique que le graphe reste connexe et est un témoin de $G$
    \\~\\
    On construit alors un surgraphe de $H'$ que l'on va plonger dans le plan. On construit le
    surgraphe $H''$ en itérant sur tout sommet $v \in V$, et en ajoutant une arrête entre chaque paire $u_1, u_2$
    voisine de $v$ adjacente à une même face, arrête prenant la forme d'un chemin dans cette face. Si $v$ est
    un sommet extérieur dans $H'$, et que l'on considère $2$ de ses voisins dans $H'$ eux aussi extérieurs,
    alors on choisira le chemin de telle sorte à ce que $v$ devienne intérieur dans $H''$.
    Ainsi tout les élements de $U'$ deviennent de degré au moins $3$. $G$ étant $3$-connexe, on peut alors voir que $H''$
    le devient également (on ne peut plus isoler de sommet de $U'$). $H''$ est également planaire et par construction,
    tout les sommets $v \in V$ sont intérieurs.
    \\~\\
    \textbf{(merde faut montrer que les adjacences restent cohérentes aussi).}
    On construit à présent la carte : pour $v \in V$, la région associée à $v$ est l'adhérence de la face dans laquelle se trouve
    le point $v$ dans $H'' - v$. Cette face est bien homéomorphe à un disque, comme $H'' - v$ est $2$-connexe et qu'alors
    cette dernière est bordée par un cycle (voir ...), le théorème de Jordan-Schönflies nous donne ce que l'on veut.
    Cet ensemble de région correspond alors à l'intérieur du cycle de $H''$
    séparant la face non bornée des autres (comme $H''$ est $2$-connexe). Ainsi, pour les mêmes raisons, la face non bornée, vue dans $\mathbb{S}^2$,
    est homéomorphe à un disque. On ajoute alors un sommet à $G$ correspondant à cette face et à ses adjacences dans cette carte.
    La carte est bien complète par construction

\end{proof}

\begin{def*}[Arrête sans trou]
    On dit qu'une arrête $xy$ d'un graphe $G$ $3$-connexe est sans trou si et seulement si pour toute paire de cliques $K^1, K^2$, voisines toutes deux
    de $x$ et $y$, si $G - (K^1 \cup K^2 \cup \{x,y\})$ n'est pas connexe, alors $G - (K^1 \cup K^2 \cup S)$ ne l'est pas non plus pour
    un certain $S \subsetneq \{x, y\}$
\end{def*}

La définition est motivée par... le dessin qui n'est pas là
\\~\\
On considèrera par la suite que tout les graphes $G$ sont de carte, $3$-connexe, et admettent une carte complète

\begin{lem}\label{contrSanstrou}
    Soit $xy \in E$ tel que dans une certaine carte complète de $G$, $x$ et $y$ se rencontrent pas qu'en un seul point.
    Si $xy$ est sans trou, alors $G / xy$ admet une carte complète et reste $3$-connexe
\end{lem}

\begin{proof}
    On se donne une carte complète de $G$ vérifiant ces conditions. Supposons par l'absurde que $x$ et $y$ ne se rencontrent pas selon un chemin.
    On raisonne selon la nature de $x \cap y$
    \\~\\
    Supposons cet ensemble discret, donc fini car compact. Notons alors ses points
    $u_1, ..., u_k$, $k \geq 2$ dans l'ordre cyclique selon la frontière de $x$. On construit alors $H$ un témoin de $G$ comme suit :
    on place $x$ et $y$ dans leurs régions respectives, $u_1, ..., u_k$ à leur position respectives et on relie $x$ et $y$ aux
    $u_i$ par des chemins dans les régions $x$ et $y$. La carte étant complète, en suivant le chemin de $u_1$ à $u_2$ le long de la frontière de
    $x$, on construit un chemin dans $G$ de la même manière que dans le lemme \ref{cycleCompl}. On place alors les régions et points
    de rencontre comme discuté dans le lemme \ref{cycleCompl} dans $H$. Ainsi, l'intérieur du cycle $x, u_1, y, u_{k}$ dans $H$
    contient au moins un sommet de $V$. La carte étant complète, il existe des régions, extérieures aux cycles
    précedemment décrits, contenant les points $u_1$ et $u_k$. Les régions, distinctes de $x$ et $y$, contenant $u_1$ forment une clique $K^1$ et celle contenant
    $u_k$ une clique $K^2$, voisines toutes deux de $x$ et $y$. En retirant les sommets de $V$ formant ces cliques du graphe $H$, on remarque
    qu'il n'est plus connexe, comme l'intérieur du cycle $x, u_1, y, u_k$ ne peut plus accéder aux sommets extérieurs à ce dernier, le seul chemin
    possible empruntant $u_1$ ou $u_k$ qui n'a plus de voisins extérieurs. Donc $G - (K^1 \cup K^2 \cup \{x,y\})$ n'est pas connexe. Or
    $G - (K^1 \cup K^2 \cup S)$, $S \subsetneq \{x, y\}$, l'est car le témoin $H$ où l'on retire uniquement les sommets adjacents à $u_1$ ou $u_k$ extérieurs
    au cycle et éventuellement soit $x$ soit $y$ est connexe par construction de ce dernier
    \\~\\
    Si cet ensemble n'est pas discret, son nombre de composantes connexes étant fini par compacité, il est alors union disjointe de chemins
    et de points.\\
    Si cet ensemble est connexe, il ne s'agit que d'un chemin, alors on peut construire une carte complète pour $G / xy$ en
    représentant $xy$ par l'intérieur du chemin obtenu en concaténant les parties disjointes des frontières de $x$ et $y$, auquel on ajoute ce chemin.
    \\~\\
    Sinon : cet ensemble ne peut contenir $2$ chemins disjoints par $3$-connexité : en effet dans ce cas $G - \{x, y\}$ n'est pas connexe.
    Donc il contient un chemin et un point n'appartenant pas à ce dernier. On choisit alors un point $u_1$ dans l'intérieur du chemin (pas à une
    de ses éxtrémités) et on note $u_2$ un point de $x \cap y$ n'appartenant pas au chemin. On construit un témoin $H$ de la même manière que précédemment.
    L'intérieur du cycle $x, u_1, y, u_2$ contient un sommet de $V$ pour les mêmes raisons qu'auparavant, sauf qu'ici $u_1$ étant dans l'intérieur
    du chemin ne peut être adjacent à ce sommet. Notons $K$ la clique formée des sommets extérieurs à ce cycle, excepté $x$ et $y$, contenant le point $u_2$.
    $G - K$ est connexe, pour les mêmes raisons que pour le cas précédent, tandis que $G - (K \cup \{x, y\})$ ne l'est pas
    \\~\\
    La $3$-connexité vient directement de la définition d'arrête sans trou
\end{proof}

\begin{lem}
    Pour tout $v \in V$, il existe $u$ un voisin de $v$ tel que $uv$ vérifie les conditions du lemme \ref{contrSanstrou}
\end{lem}

\begin{proof}
    On se donne une carte complète de $G$. La preuve du lemme \ref{cycleCompl} permet d'affirmer qu'il existe des régions $u$ adjacentes à $v$
    telles que $u \cap v$ ne soit pas discret. On suppose alors par l'absurde que toutes ces régions sont telles que $u \cap v$ ne soit
    pas connexe. Mais alors, $u \cup v$ sépare $\mathbb{S}^2$ en au moins $2$ composantes connexes, et la carte étant complète les deux sont totalement
    recouvertes par des régions. Ainsi, $v$ a des voisins tels que leur intersection n'est pas discrète dans les deux composantes connexes ainsi délimitées.
    On considère alors un voisin dans l'une de ces composantes que l'on note $C$ et on réitère le processus, en choisissant à chaque fois une composante incluse dans $C$.
    On obtient ainsi une infinité de régions distinctes ce qui est absurde car le graphe est fini.
    \\~\\
    Ainsi il existe $u$ voisin de $v$ tel que $u \cap v$ soit un chemin. On montre alors qu'il existe un tel $u$ tel que $uv$ soit sans trou.
    Bon globalement faut distinguer selon si la frontière est recouverte ou pas
\end{proof}

\begin{prop}
    Un graphe $G$ $3$-connexe admet une carte complète si et seulement si il existe une certaine arrête $xy \in E$ telle que $G / xy$ admette une carte
    complète, reste $3$-connexe, et telle que, blabla condition sur le témoin de $G / xy$, globalement $x$ et $y$ se partagent leurs voisins de manière nice 
\end{prop}

\section{Algorithmes}

La proposition d'avant est la base d'un algorithme

\end{flushleft}
\end{document}